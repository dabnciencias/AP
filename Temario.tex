\documentclass[a4paper]{article}

\usepackage[margin=1.5cm]{geometry}
\usepackage{amsmath,amsthm,amssymb,tabu}
\usepackage[spanish,es-tabla]{babel}
\decimalpoint
\usepackage[T1]{fontenc}
\usepackage[utf8]{inputenc}
\usepackage{lmodern}
\usepackage[dvipsnames]{xcolor}
\usepackage[hyphens]{url}
\usepackage{graphicx}
\graphicspath{ {images/} }
\usepackage{tcolorbox}
\usepackage{enumitem}
\setcounter{section}{-1}
\usepackage{tabularx}
\usepackage{multirow}
\usepackage{hyperref}
\usepackage{braket}
\usepackage{tikz}
\usepackage{mathrsfs}
\usetikzlibrary{cd}
\usepackage{pgfplots}
\usepackage{caption}
\usepackage{subcaption}
\usetikzlibrary{babel}

\definecolor{NARANJA}{rgb}{1,0.467,0}
\definecolor{VERDE}{rgb}{0.31,1,0}
\definecolor{AZUL}{rgb}{0,0.53,1}
\definecolor{ROJO}{rgb}{1,0,0}


\hypersetup{
    colorlinks=true,
    linkcolor=ROJO,
    filecolor=magenta,
    urlcolor=AZUL,
}
\newenvironment{theorem}[2][Theorem]{\begin{trivlist}
\item[\hskip \labelsep {\bfseries #1}\hskip \labelsep {\bfseries #2.}]}{\end{trivlist}}
\newenvironment{teorema}[2][Teorema]{\begin{trivlist}
\item[\hskip \labelsep {\bfseries #1}\hskip \labelsep {\bfseries #2.}]}{\end{trivlist}}
\newenvironment{lema}[2][Lema]{\begin{trivlist}
\item[\hskip \labelsep {\bfseries #1}\hskip \labelsep {\bfseries #2.}]}{\end{trivlist}}
\newenvironment{exercise}[2][Exercise]{\begin{trivlist}

\item[\hskip \labelsep {\bfseries #1}\hskip \labelsep {\bfseries #2.}]}{\end{trivlist}}
\newenvironment{problem}[2][Problem]{\begin{trivlist}
\item[\hskip \labelsep {\bfseries #1}\hskip \labelsep {\bfseries #2.}]}{\end{trivlist}}
\newenvironment{question}[2][Question]{\begin{trivlist}
\item[\hskip \labelsep {\bfseries #1}\hskip \labelsep {\bfseries #2.}]}{\end{trivlist}}
\newenvironment{corollary}[2][Corollary]{\begin{trivlist}
\item[\hskip \labelsep {\bfseries #1}\hskip \labelsep {\bfseries #2.}]}{\end{trivlist}}
\newenvironment{corolario}[2][Corolario]{\begin{trivlist}
\item[\hskip \labelsep {\bfseries #1}]}{\end{trivlist}}
\newenvironment{solution}{\begin{proof}[Solution]}{\end{proof}}

\pgfplotsset{compat=1.15}

\begin{document}
\title{Seminario de Aplicaciones de Cómputo \\ Animación Programática (con Python, \\ Manim y Programación Orientada a Objetos) }
\author{M. en C. Diego Alberto Barceló Nieves \\ Facultad de Ciencias, Universidad Nacional Autónoma de México}
\date{}
\maketitle

\textbf{Nota} Dado que este no es un curso introductorio de programación, al elaborar el siguiente temario asumimos que quienes tomarán el curso tienen algún tipo de experiencia previa programando; sin embargo, no asumimos que tengan familiaridad con el lenguaje de programación \hyperlink{https://www.python.org/}{Python}, la librería \hyperlink{https://www.manim.community/}{Manim}, ni el paradigma de la \hyperlink{https://developer.mozilla.org/es/docs/Glossary/OOP}{Programación Orientada a Objetos}.

\section*{Objetivo general} \label{Sec: Objetivo general}

Aprender a visualizar y animar conceptos matemáticos de forma altamente precisa utilizando herramientas de programación.

\section*{Objetivos específicos} \label{Sec: Objetivos específicos}

\begin{enumerate}

    \item Aprender el uso básico del lenguaje de programación Python.

    \item Entender el paradigma de Programación Orientada a Objetos y poder aplicarlo, creando nuevas clases o atributos cuando sea necesario.

    \item Entender los principios de diseño del \emph{software} Manim.

    \item Aprender a realizar animaciones precisas (programadas mediante un \emph{script} de Python y usando la librería Manim) para representar conceptos en áreas fundamentales de las matemáticas tales como geometría analítica, álgebra lineal, cálculo diferencial e integral y ecuaciones diferenciales, entre otras.

    \item Aprender a instalar, utilizar, modificar y contribuir a proyectos de \emph{software} de código abierto (\hyperlink{https://github.com/ManimCommunity/manim}{como Manim}).
\end{enumerate}

\setcounter{section}{-1}

\section{Introducción a Python y Programación Orientada a Objetos} \label{Sec: Introducción a Python y Programación Orientada a Objetos} 

\begin{enumerate}[label=\arabic*.]

    \item Manejo de terminal virtual e instalación y uso básico de Python, Jupyter y Git.
    \begin{enumerate}[label=1.\arabic*]
    
        \item Navegación en la terminal virtual y variables de ambiente.

        \item Manipulación de archivos y carpetas.

        \item Administradores de paquetes, instalaciones y la variable \texttt{PATH}.

        \item REPL de Python y \emph{notebooks} interactivos de Jupyter.

        \item Sincronización de repositorios con Git.
    \end{enumerate}

    \item Sintáxis y tipos de datos básicos de Python.
    \begin{enumerate}[label=2.\arabic*]
    
        \item Operadores lógicos y tipos de datos Booleanos.

        \item Operadores aritméticos y tipos de datos numéricos.

        \item Tipos de datos de texto.

        \item Variables y constantes.

        \item Declaraciones condicionales y ciclos iterativos.

        \item Listas y diccionarios.
    \end{enumerate}

    \item Funciones.
    \begin{enumerate}[label=3.\arabic*]

        \item Definición de funciones.

        \item Pase por valor y pase por referencia.

        \item Funciones de \emph{casting}.
    
        \item Funciones recursivas.

        \item Funciones anónimas.

        \item Funciones con cantidades variables de parámetros.
    \end{enumerate}

    \item Clases y objetos.
    \begin{enumerate}[label=4.\arabic*]

        \item Parámetros y atributos.
    
        \item Métodos especiales.

        \item Clases concretas y abstractas.
    \end{enumerate}

    \item Subclases.
    \begin{enumerate}[label=5.\arabic*]

        \item Herencia simple y múltiple.
    
        \item Polimorfismo.
    \end{enumerate}
\end{enumerate}

\section{Introducción a Manim} \label{Sec: Introducción a Manim}

\begin{enumerate}

    \item Instalación y uso básico de Manim.

    \item Introducción a las clases fundamentales.
   \begin{enumerate}[label=2.\arabic*]
       \item Objetos animables con \href{https://docs.manim.community/en/v0.16.0.post0/reference/manim.mobject.mobject.Mobject.html}{\texttt{Mobject}}.

       \item Animación de escenas con \href{https://docs.manim.community/en/v0.16.0.post0/reference/manim.scene.scene.Scene.html}{\texttt{Scene}}.

       \item Manejo de cámaras con \href{https://docs.manim.community/en/v0.16.0.post0/reference/manim.camera.camera.Camera.html?highlight=Camera}{\texttt{Camera}}.

       \item Animaciones con \href{https://docs.manim.community/en/v0.16.0.post0/reference/manim.animation.animation.Animation.html#}{\texttt{Animation}}.
   \end{enumerate}

    \item Objetos animables vectorizados y grupos de objetos animables vectorizados.
    \begin{enumerate}[label=3.\arabic*]

        \item La clase \texttt{Mobject -> \ \href{https://docs.manim.community/en/v0.16.0.post0/reference/manim.mobject.types.vectorized_mobject.VMobject.html?highlight=VMobject}{VMobject}}\footnote{Utilizamos el formato \texttt{A -> \ B} para indicar que \texttt{B} es una subclase directa de \texttt{A} en el sentido de la Programación Orientada a Objetos.}.

        \item Ejemplos de VMobjects.
    
        \item La subclase \texttt{VMobject -> \ \href{https://docs.manim.community/en/v0.16.0/reference/manim.mobject.types.vectorized_mobject.VGroup.html?highlight=VGroup}{VGroup}}.

        \item La subclase \texttt{Vmobject -> \ \href{https://docs.manim.community/en/v0.16.0.post0/reference/manim.mobject.svg.svg_mobject.SVGMobject.html}{SVGMobject}}.
    \end{enumerate}

    \item Escritura de texto y \LaTeX.
    \begin{enumerate}[label=4.\arabic*]

        \item La clase \texttt{SVGMobject -> \ \href{https://docs.manim.community/en/v0.16.0.post0/reference/manim.mobject.text.text_mobject.Text.html?highlight=Text}{Text}}.

        \item La clase \texttt{SVGMobject -> \ SingleStringMathTex -> \ \href{https://docs.manim.community/en/v0.16.0.post0/reference/manim.mobject.text.tex_mobject.MathTex.html?highlight=MathTex}{MathTex}}.
    \end{enumerate}

    \item Configuración.
    \begin{enumerate}[label=5.\arabic*]

        \item Diccionarios de configuración.

        \item La clase \href{https://docs.manim.community/en/v0.16.0.post0/reference/manim._config.utils.ManimConfig.html?highlight=ManimConfig}{\texttt{ManimConfig}}.

        \item Archivos de configuración.
    \end{enumerate}

\end{enumerate}

\section{Escenas geométricas en dos dimensiones} \label{Sec: Escenas geométricas en dos dimensiones}

\begin{enumerate}

    \item Planos cartesianos, puntos y líneas.
    \begin{enumerate}[label=5.\arabic*]

        \item Planos cartesianos con \href{https://docs.manim.community/en/v0.16.0.post0/reference/manim.mobject.graphing.coordinate_systems.NumberPlane.html?highlight=NumberPlane}{\texttt{NumberPlane}}.

        \item Puntos con \href{https://docs.manim.community/en/v0.16.0.post0/reference/manim.mobject.geometry.arc.Dot.html?highlight=Dot}{\texttt{Dot}}.

        \item Líneas rectas y curvas con \href{https://docs.manim.community/en/v0.16.0.post0/reference/manim.mobject.geometry.line.Line.html#manim.mobject.geometry.line.Line}{\texttt{Line}} y \href{https://docs.manim.community/en/v0.16.0.post0/reference/manim.mobject.geometry.arc.Arc.html?highlight=Arc}{\texttt{Arc}}.
    \end{enumerate}

    \item Figuras geométricas en dos dimensiones. (\texttt{VMobject -> \ Polygram -> \ \href{https://docs.manim.community/en/v0.16.0.post0/reference/manim.mobject.geometry.polygram.Polygon.html?highlight=Polygon}{Polygon}})
    \begin{enumerate}[label=2.\arabic*]

        \item La clase \texttt{VMobject -> \ \href{https://docs.manim.community/en/v0.16.0.post0/reference/manim.mobject.geometry.polygram.Polygram.html?highlight=polygram}{Polygram}} y la subclase \texttt{Polygram -> \ \href{https://docs.manim.community/en/v0.16.0.post0/reference/manim.mobject.geometry.polygram.Polygon.html?highlight=polygon}{Polygon}}.

        \item Las subclases \texttt{Polygram -> \ \href{https://docs.manim.community/en/v0.16.0.post0/reference/manim.mobject.geometry.polygram.RegularPolygram.html?highlight=RegularPolygram}{RegularPolygram}} y \texttt{RegularPolygram -> \ \href{https://docs.manim.community/en/v0.16.0.post0/reference/manim.mobject.geometry.polygram.RegularPolygon.html?highlight=RegularPolygon}{RegularPolygon}}.
    \end{enumerate}

    \item Sistemas coordenados.
    \begin{enumerate}[label=3.\arabic*]

        \item La clase \texttt{Line -> \ \href{https://docs.manim.community/en/v0.16.0.post0/reference/manim.mobject.graphing.number_line.NumberLine.html?highlight=NumberLine}{NumberLine}}.

        \item La clase \texttt{\href{https://docs.manim.community/en/v0.16.0.post0/reference/manim.mobject.graphing.coordinate_systems.NumberPlane.html?highlight=NumberPlane}{NumberPlane}} y la superclase \texttt{NumberPlane <- \href{https://docs.manim.community/en/v0.16.0.post0/reference/manim.mobject.graphing.coordinate_systems.Axes.html?highlight=Axes}{Axes}}.

        \item La subclase \texttt{NumberPlane -> \ \href{https://docs.manim.community/en/v0.16.0.post0/reference/manim.mobject.graphing.coordinate_systems.ComplexPlane.html?highlight=ComplexPlane}{ComplexPlane}}.
    \end{enumerate}

    \item Curvas y gráficas.
    \begin{enumerate}[label=4.\arabic*]

        \item Curvas con \href{https://docs.manim.community/en/v0.16.0.post0/reference/manim.mobject.graphing.functions.ParametricFunction.html#manim.mobject.graphing.functions.ParametricFunction}{\texttt{ParametricFunction}}.

        \item Gráficas con \href{https://docs.manim.community/en/v0.16.0.post0/reference/manim.mobject.graphing.functions.FunctionGraph.html#manim.mobject.graphing.functions.FunctionGraph}{\texttt{FunctionGraph}}.
    \end{enumerate}
\end{enumerate}

\section{Cámaras móviles en dos dimensiones} \label{Sec: Cámaras móviles en dos dimensiones}

\begin{enumerate}

    \item Movimiento de cámara con las clases \texttt{Scene -> \ \href{https://docs.manim.community/en/v0.16.0.post0/reference/manim.scene.moving_camera_scene.MovingCameraScene.html?highlight=MovingCameraScene}{MovingCameraScene}} y \texttt{Camera -> \ \href{https://docs.manim.community/en/v0.16.0.post0/reference/manim.camera.moving_camera.MovingCamera.html?highlight=MovingCamera}{MovingCamera}}.

    \item Acercamiento y alejamiento de cámara con la subclase \texttt{MovingCameraScene -> \ \href{https://docs.manim.community/en/v0.16.0.post0/reference/manim.scene.zoomed_scene.ZoomedScene.html?highlight=ZoomedScene}{ZoomedScene}}.

    \item Manejo de múltiples cámaras con la subclase \texttt{MovingCamera -> \ \href{https://docs.manim.community/en/v0.16.0.post0/reference/manim.camera.multi_camera.MultiCamera.html?highlight=MultiCamera}{MultiCamera}}.
\end{enumerate}

\section{Escenas vectoriales en dos dimensiones} \label{Sec: Escenas vectoriales en dos dimensiones}

\begin{enumerate}

    \item Flechas y vectores flecha.
    \begin{enumerate}[label=1.\arabic*]

        \item Flechas con la clase \texttt{Line -> \ \href{https://docs.manim.community/en/v0.16.0.post0/reference/manim.mobject.geometry.line.Arrow.html?highlight=Arrow}{Arrow}}.

        \item Vectores flecha con la subclase \texttt{Arrow -> \ \href{https://docs.manim.community/en/v0.16.0.post0/reference/manim.mobject.geometry.line.Vector.html?highlight=Vector}{Vector}}.
    \end{enumerate}

    \item Escenas vectoriales con \texttt{Scene -> \ \href{https://docs.manim.community/en/v0.16.0.post0/reference/manim.scene.vector_space_scene.VectorScene.html?highlight=VectorScene}{VectorScene}}.

    \item Representación de transformaciones lineales con \texttt{VectorScene -> \ \href{https://docs.manim.community/en/v0.16.0.post0/reference/manim.scene.vector_space_scene.LinearTransformationScene.html?highlight=LinearTransformationScene}{LinearTransformationScene}}.

    \item Campos vectoriales con \href{https://docs.manim.community/en/v0.16.0.post0/reference/manim.mobject.vector_field.ArrowVectorField.html?highlight=ArrowVectorField}{\texttt{ArrowVectorField}} y \href{https://docs.manim.community/en/v0.16.0.post0/reference/manim.mobject.vector_field.StreamLines.html?highlight=StreamLines}{\texttt{StreamLines}}.
\end{enumerate}

\section{Escenas en tres dimensiones} \label{Sec: Escenas en tres dimensiones}

\begin{enumerate}

    \item Escena y cámara tridimensionales con \texttt{Scene -> \ \href{https://docs.manim.community/en/v0.16.0.post0/reference/manim.scene.three_d_scene.ThreeDScene.html?highlight=ThreeDScene}{ThreeDScene}} y \texttt{Camera -> \ \href{https://docs.manim.community/en/v0.16.0.post0/reference/manim.camera.three_d_camera.ThreeDCamera.html?highlight=ThreeDCamera}{ThreeDCamera}}.

    \item Puntos, curvas y superficies en 3D.
    \begin{enumerate}[label=2.\arabic*]

        \item Puntos en tridimensionales con \href{https://docs.manim.community/en/v0.16.0.post0/reference/manim.mobject.three_d.three_dimensions.Dot3D.html?highlight=Dot3D}{\texttt{Dot3D}}.

        \item Curvas en tres dimensiones con \href{https://docs.manim.community/en/v0.16.0.post0/reference/manim.mobject.graphing.functions.ParametricFunction.html?highlight=ParametricFunction#threedparametricspring}{\texttt{ParametricFunction}}.

        \item Superficies con \href{https://docs.manim.community/en/v0.16.0.post0/reference/manim.mobject.three_d.three_dimensions.Surface.html?highlight=Curve3D}{\texttt{Surface}}.
    \end{enumerate}

    \item Líneas y flechas en 3D con \href{https://docs.manim.community/en/v0.16.0.post0/reference/manim.mobject.three_d.three_dimensions.Line3D.html?highlight=Line3D}{\texttt{Lined3D}} y \href{https://docs.manim.community/en/v0.16.0.post0/reference/manim.mobject.three_d.three_dimensions.Arrow3D.html?highlight=Arrow3D}{\texttt{Arrow3D}}.

    \item Texto en 3D.
\end{enumerate}

\section{Animaciones especiales} \label{Sec: Animaciones especiales}

\begin{enumerate}

    \item \href{https://docs.manim.community/en/v0.16.0.post0/reference/manim.animation.transform.html#module-manim.animation.transform}{Transformaciones}. 

    \item \href{https://docs.manim.community/en/v0.16.0.post0/reference/manim.animation.updaters.html#module-manim.animation.updaters}{Actualizadores}. ()

    \item Animación de grupos. (\texttt{Animation -> \ AnimationGroup -> \ TransformMatchingAbstractBase} \texttt{\{-> \\ \href{https://docs.manim.community/en/v0.16.0.post0/reference/manim.animation.transform_matching_parts.TransformMatchingShapes.html?highlight=TransformMatchingShapes}{TransformMatchingShapes}} y \texttt{-> \ \href{https://docs.manim.community/en/v0.16.0.post0/reference/manim.animation.transform_matching_parts.TransformMatchingTex.html?highlight=TransformMatchingTex}{TransformMatchingTex}\}})

    \item Representaciones de algoritmos.

\end{enumerate}

\section*{Bibliografía básica} \label{Sec: Bibliografía básica}

\begin{enumerate}

    \item \href{https://docs.manim.community/en/stable/index.html}{Documentación de Manim Community Edition.}

    %\item Repositorio de GitHub \href{}{dabnciencias/AP}.

    \item Deitel y Deitel,  \emph{Intro to Python for Computer Science and Data Science} (2021).
\end{enumerate}

\section*{Bibliografía complementaria} \label{Sec: Bibliografía complementaria}

\begin{enumerate}

\item Lista de reproducción \href{https://www.youtube.com/watch?v=-VJ7h8-GbHU&list=PL91agCMqt_mfPlTgR8zmguMZIpGV0Jflj&ab_channel=Animathica}{\emph{Seminario Animathica (de Animación Programática)}} del canal de YouTube ''\href{https://www.youtube.com/channel/UCzkyH2bxpesubzc87VxqDiA}{Animathica}'' y repositorio de GitHub \href{https://github.com/animathica/seminario}{animathica/seminario}.

\item Lista de reproducción \href{https://www.youtube.com/playlist?list=PLcjmqHFN9VeMC9znnNiRMv3nqZv-bU9Fo}{\emph{Tutorial de Manim en español}} del canal de YouTube ``\href{https://www.youtube.com/channel/UCmFww1CGIFsvujZ0zNzcLQw}{El teorema de Beethoven}''.

    \item \href{https://www.python.org/doc/}{Documentación de Python}.

    \item Página web \href{https://www.geeksforgeeks.org/python-programming-language/}{GeeksforGeeks}.

    \item \href{https://docs.jupyter.org/en/latest/}{Documentación de Jupyter}.
\end{enumerate}


\end{document}
